%
% Copyright � 2015 Peeter Joot.  All Rights Reserved.
% Licenced as described in the file LICENSE under the root directory of this GIT repository.
%

%
%\chapter{Preface}
% this suppresses an explicit chapter number for the preface.
\chapter*{Preface}%\normalsize
  \thispagestyle{empty}
  \addcontentsline{toc}{chapter}{Preface}

This document was produced while taking the Spring 2016, University of Toronto Microwave and Millimeter-Wave Techniques (aka Microwave Circuits) course (ECE1236H), taught by Prof.\ G. V. Eleftheriades.
\paragraph{Course Syllabus}
This course outlines the principles of designing modern microwave and RF circuits.
\begin{itemize}
\item Signal-integrity issues in high-speed digital circuits.
\item The wave equation.
\item Ideal transmission lines.
\item Transients on transmission-lines.
\item Planar transmission lines and introduction to MMIC's.
\item Designing with scattering parameters.
\item Planar power dividers.
\item Directional couplers.
\item Microwave filters.
\item Solid-state microwave amplifiers.
\item Noise.
\item Diode-mixers.
\item RF receiver chains.
\item Oscillators.
\end{itemize}

The text for this course was \citep{pozar2009microwave}.  It was not a required text, and the course was taught off of slides.
\withproblemsetsMessage{
\textcolor{Maroon}{
\textit{THIS DOCUMENT IS REDACTED.  THE PROBLEM SET SOLUTIONS AND ASSOCIATED MATHEMATICA CODE IS NOT VISIBLE.  PLEASE EMAIL ME FOR THE FULL VERSION IF YOU ARE NOT TAKING ECE1236.}
}
}

\paragraph{This document contains:}
\begin{itemize}
\item Lecture notes (or transcriptions from the class slides when they went too fast).
\item Personal notes exploring auxiliary details.
\item Worked practice problems.
\ifthenelse{\boolean{redacted}}%
{%
\item Links to Matlab, Mathematica, and Julia notebooks associated with the course material and problems (but not problem sets).
}%
{
\item Assigned problems.%
\item Links to Matlab, Mathematica, and Julia notebooks associated with problems and course material.%
}
\end{itemize}

My thanks go to Professor Eleftheriades for teaching this course.

Peeter Joot  \quad peeterjoot@pm.me
