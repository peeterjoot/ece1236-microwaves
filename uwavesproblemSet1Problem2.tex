%
% Copyright � 2016 Peeter Joot.  All Rights Reserved.
% Licenced as described in the file LICENSE under the root directory of this GIT repository.
%
\makeproblem{Low-loss transmission line with non-magnetic dielectric.}{uwaves:problemSet1:2}{
\index{transmission line!low loss}
%\makesubproblem{}{uwaves:problemSet1:2a}
A transmission line is filled with a non-magnetic dielectric of \( \epsilon_r = 2.5\). The line has a capacitance per unit length of \( C = 200 \,\si{pF/m}\) and a resistance per unit length of \( R = 2 \Omega/\si{m} \).
Calculate the corresponding phase velocity, characteristic impedance and attenuation constant \( \alpha \) (assume \(G = 0\)).
} % makeproblem

\makeanswer{uwaves:problemSet1:2}{

%\makeSubAnswer{}{uwaves:problemSet1:2a}

The impedance of the line is

\begin{dmath}\label{eqn:uwavesproblemSet1Problem2:20}
Z_0 = \sqrt{\frac{ R + j \omega L }{ G + j \omega C }},
\end{dmath}

we know \( R, C, G \), but not \( L \).  We'd found that

\begin{dmath}\label{eqn:uwavesproblemSet1Problem2:40}
v_\phi = \inv{\sqrt{L C}},
\end{dmath}

for lossless, low-loss, and distortionless lines.  If we assume that is the case here too, and

\begin{dmath}\label{eqn:uwavesproblemSet1Problem2:60}
v_\phi
= \frac{c}{\sqrt{\epsilon_r}}
= \frac{3 \times 10^8}{\sqrt{2.5}}
= 1.9 \times 10^8 \,\si{m/s},
\end{dmath}

we have

\begin{dmath}\label{eqn:uwavesproblemSet1Problem2:80}
L
= \inv{v_\phi^2 C}
= \frac{1}{(1.9 \times 10^8)^2 (200 \times 10^{-12})} \,\si{H/m}
= 14 \,\si{ n H/m }.
\end{dmath}

For frequencies in the \si{GHz} range are the low-loss conditions satisfied?  With \( G = 0 \), the condition \( G \ll \omega C \) is clearly true.  How about \( R \ll \omega L \)?  For \( f \) in \si{GHz} we have

\begin{dmath}\label{eqn:uwavesproblemSet1Problem2:300}
\omega L
= 2( 14 ) \pi f
= 88 f.
\end{dmath}

We have \( 2 < 88 f \), so the low-loss approximation is definitely valid for frequencies above 1 \si{GHz}.

Without such approximations, our impedance expression is a bit of an ugly beast

\begin{dmath}\label{eqn:uwavesproblemSet1Problem2:100}
Z_0
= \sqrt{\frac{ R + j \omega L }{ G + j \omega C }}
= \sqrt{\frac{ 2 + j \omega 14 \times 10^{-9} }{ j \omega 200 \times 10^{-12} }} \Omega.
\end{dmath}

The low-loss approximation for the impedence, with \( R = 0 \), is

\begin{equation}\label{eqn:uwavesproblemSet1Problem2:160}
Z_0 = 26.35.
\end{equation}

Compare this to the exact expansion of \cref{eqn:uwavesproblemSet1Problem2:100} at frequencies starting at 1 \,\si{GHz}

\begin{equation}\label{eqn:uwavesproblemSet1Problem2:140}
\begin{aligned}
Z_0(1 \,\si{GHz}) &= 26.35 \phase{ \ang{-0.06566} } \\
Z_0(2 \,\si{GHz}) &= 26.35 \phase{ \ang{-0.03283} } \\
Z_0(3 \,\si{GHz}) &= 26.35 \phase{ \ang{-0.02189} } \\
Z_0(10 \,\si{GHz}) &= 26.35 \phase{ \ang{-0.006566} } \\
\end{aligned}
\end{equation}

At all of these frequencies, the impedance matches the low-loss approximation.

For the attenuation factor first note that
\begin{dmath}\label{eqn:uwavesproblemSet1Problem2:120}
\gamma
= \lr{ \lr{ 2 + j \omega 1.4 \times 10^{-9} } \lr{ j \omega 200 \times 10^{-12} } }^{1/2}
= \lr{ \lr{ 2 + j 2 \pi f 1.4 } \lr{ j 0.4 \pi f } }^{1/2}
= f \sqrt{ 0.4 \pi } \lr{ (2/f) j - 2.8 \pi }^{1/2},
\end{dmath}

where \( f \) is in \si{GHz}.  We see that for \( f \in [1, 10] \), this \( \gamma \) is close to purely imaginary, as is also the case for \( R = 0 \).  Specifically

\begin{equation}\label{eqn:uwavesproblemSet1Problem2:180}
\begin{aligned}
\gamma(1  \,\si{GHz})/f &= 33.12 \phase{ \ang{89.93} } \\
\gamma(2  \,\si{GHz})/f &= 33.12 \phase{ \ang{89.97} } \\
\gamma(3  \,\si{GHz})/f &= 33.12 \phase{ \ang{89.98} } \\
\gamma(10  \,\si{GHz})/f &= 33.12 \phase{ \ang{89.99} } \\
\end{aligned}
\end{equation}

We want the real component of this.
%Whereas for \( R = 0 \) we have
%
%\begin{equation}\label{eqn:uwavesproblemSet1Problem2:200}
%\gamma/f = 3.312 j.
%\end{equation}
%
%It appears that \( \alpha \sim 0 \) for this system in the \si{GHz} range.
For the same frequencies, some specific values of the real part, are

\begin{equation}\label{eqn:uwavesproblemSet1Problem2:220}
\begin{aligned}
\alpha(1  \,\si{GHz}) &= 0.037947307 \\
\alpha(2  \,\si{GHz}) &= 0.037947326 \\
\alpha(3  \,\si{GHz}) &= 0.037947329 \\
\alpha(10  \,\si{GHz}) &= 0.037947332.
\end{aligned}
\end{equation}

How much attenuation is this, relative to the wavelength in question?
Even for lower frequencies, where the wavelength is larger, there is essentially zero attenuation over one wavelength.
For the same respective frequencies we have

\begin{equation}\label{eqn:uwavesproblemSet1Problem2:240}
\begin{aligned}
e^{-\lambda \alpha} &= 0.9928 \\
e^{-\lambda \alpha} &= 0.9964 \\
e^{-\lambda \alpha} &= 0.9976 \\
e^{-\lambda \alpha} &= 0.9993.
\end{aligned}
\end{equation}

Let's see how the
low-loss line approximation fares for both the real (\(\alpha\)) and imaginary (\(\beta\)) components of \( \gamma \)? For all of these frequencies, we have

\begin{equation}\label{eqn:uwavesproblemSet1Problem2:260}
\gamma = 0.377 + 0.5305 j \omega.
\end{equation}

which compares exactly with the low-loss approximations using \( Z_0 = \sqrt{L/C}, \alpha = (R/Z_0 + G Z_0)/2, \beta = \omega \sqrt{LC} \).

The numerical calculations made for this problem can be found in \nbref{ps1:ps1_2.jl}.
}
