%
% Copyright � 2016 Peeter Joot.  All Rights Reserved.
% Licenced as described in the file LICENSE under the root directory of this GIT repository.
%
\input{../latex/blogpost.tex}
\renewcommand{\basename}{uwaves2}
\renewcommand{\dirname}{notes/ece1236/}
\newcommand{\keywords}{ECE1236H}
\input{../latex/peeter_prologue_print2.tex}

%\usepackage{ece1236}
\usepackage{peeters_braket}
%\usepackage{peeters_layout_exercise}
\usepackage{peeters_figures}
\usepackage{enumerate}
\usepackage{siunitx}
\usepackage{mathtools}
\usepackage{macros_cal}
\usepackage{macros_bm}

\beginArtNoToc
\generatetitle{ECE1236H Microwave and Millimeter-Wave Techniques.  Lecture 2: Wave equation and Poynting theorem.  Taught by Prof.\ G.V. Eleftheriades}
%\chapter{Wave equation and Poynting theorem}
\label{chap:uwaves2}

\paragraph{Disclaimer}

Peeter's lecture notes from class.  These may be incoherent and rough.

These are notes for the UofT course ECE1236H, Microwave and Millimeter-Wave Techniques, taught by Prof. G.V. Eleftheriades, covering \textchapref{{1}} \citep{pozar2009microwave} content.

\EndArticle
%\EndNoBibArticle
