%
% Copyright © 2016 Peeter Joot.  All Rights Reserved.
% Licenced as described in the file LICENSE under the root directory of this GIT repository.
%
\section{Plane waves}
\index{plane wave}

In simple media with no sources

\begin{subequations}
\label{eqn:uwavesLecture2:20}
\begin{equation}\label{eqn:uwavesLecture2:40}
\spacegrad \cross \BE = - \PD{t}{\BB} = - \mu \PD{t}{\BH},
\end{equation}
\begin{equation}\label{eqn:uwavesLecture2:60}
\spacegrad \cdot \BH = 0.
\end{equation}
\begin{equation}\label{eqn:uwavesLecture2:80}
\spacegrad \cross \BH =
\BJ + \PD{t}{\BD} =
\lr{ \sigma + \epsilon \PD{t}{} } \BE
\end{equation}
\begin{equation}\label{eqn:uwavesLecture2:100}
\spacegrad \cdot \BE = \frac{\rho}{\epsilon} = 0
\end{equation}
\end{subequations}

Here \( \BJ = \sigma \BE \) is considered to be an induced current, and the \( \rho/\epsilon \) term vanishes since we have no sources.

From Faraday's law \cref{eqn:uwavesLecture2:40}

\begin{dmath}\label{eqn:uwavesLecture2:120}
\spacegrad \cross \lr{ \spacegrad \cross \BE }
= - \mu \PD{t}{} \lr { \spacegrad \cross \BH }
= - \mu \PD{t}{} \lr { \sigma + \epsilon \PD{t}{} } \BE
= - \mu \sigma \PD{t}{\BE} - \epsilon \mu \PDSq{t}{\BE}.
\end{dmath}

We can also use the vector identity

\begin{dmath}\label{eqn:uwavesLecture2:140}
\spacegrad \cross \lr{ \spacegrad \cross \BF } = \spacegrad \lr{ \spacegrad \cdot \BF } - \spacegrad^2 \BF,
\end{dmath}

so
\begin{dmath}\label{eqn:uwavesLecture2:160}
\spacegrad \cross \lr{ \spacegrad \cross \BE }
=
\spacegrad \lr{ \cancel{\spacegrad \cdot \BE} } - \spacegrad^2 \BE,
\end{dmath}

or
\begin{dmath}\label{eqn:uwavesLecture2:180}
0 = \spacegrad^2 \BE - \mu \sigma \PD{t}{\BE} - \epsilon \mu \PDSq{t}{\BE}.
\end{dmath}

The \( \mu \sigma \PDi{t}{\BE} \) term is a damping contribution.

Similarly for the magnetic field
\begin{dmath}\label{eqn:uwavesLecture2:200}
0 = \spacegrad^2 \BH - \mu \sigma \PD{t}{\BH} - \epsilon \mu \PDSq{t}{\BH}.
\end{dmath}

This is the wave or Helmholtz equation.
\index{Helmholtz equation}

Note that for the no loss condition \( \sigma = 0 \), we have the undamped wave equations
\begin{subequations}
\label{eqn:uwavesLecture2:220}
\begin{dmath}\label{eqn:uwavesLecture2:240}
\spacegrad^2 \BE = \epsilon \mu \PDSq{t}{\BE}.
\end{dmath}
\begin{dmath}\label{eqn:uwavesLecture2:260}
\spacegrad^2 \BH = \epsilon \mu \PDSq{t}{\BH}.
\end{dmath}
\end{subequations}

This is one wave equation for each of \( E_x, E_y, E_z, H_x, H_y\) and \( H_z \).

\paragraph{Propagation in one direction}
\index{propagation!one direction}

Consider the undamped propagation along the z direction of \( E_x(z, t) \), which must satisfy
\begin{dmath}\label{eqn:uwavesLecture2:400}
\spacegrad^2 E_x = \epsilon \mu \PDSq{t}{E_x},
\end{dmath}

or
\begin{dmath}\label{eqn:uwavesLecture2:420}
\PDSq{z}{E_x} = \epsilon \mu \PDSq{t}{E_x}.
\end{dmath}

This is analogous to the voltage transmission line equation
\begin{dmath}\label{eqn:uwavesLecture2:440}
\PDSq{z}{V} = L C \PDSq{t}{V},
\end{dmath}

With solution

\begin{equation}\label{eqn:uwavesLecture2:280}
V(z, t) =
V_0^{+} f(z - v_\phi t)
+
V_0^{-} f(z + v_\phi t),
\end{equation}

where \( v_\phi = \ifrac{1}{\sqrt{L C}} \) is the phase velocity, \( f(z - v_\phi t)\) is the forward wave, and \( f(z + v_\phi t) \) is the reflected wave.  By analogy the solution to the 1D wave equation is

\begin{equation}\label{eqn:uwavesLecture2:300}
E_x(z, t) =
E_0^{+} f(z - v_\phi t)
+
E_0^{-} f(z + v_\phi t),
\end{equation}

where \( v_\phi = \ifrac{1}{\sqrt{\epsilon \mu}} \) is the phase velocity.

For myself, the transmission line equation is something that I only encountered in this class (in a later lecture), so this analogy isn't a great one.  That said, wave equation solutions are very familiar, so not much motivation for the structure of the solution is really required.

\paragraph{More rigorous derivation of the 1D solution}

Suppose that we assume the form of the solution is

\begin{equation}\label{eqn:uwavesLecture2:320}
\BE = \BE_0 f(z - v_\phi t ),
\end{equation}

where \( \BE_0 \) is a constant vector, so this describes a wave propagation in the z-direction, but without a-priori knowledge of the direction of this vector in space.

For source free media, we have

\begin{dmath}\label{eqn:uwavesLecture2:340}
\spacegrad \cdot ( \BE_0 f )
= \spacegrad f \cdot \BE_0 + f \cancel{ \spacegrad \cdot \BE }
= (\zcap f') \cdot \BE_0.
\end{dmath}

Because \( \BE_0 \cdot \zcap = 0 \), the vector \( \BE_0 \) is perpendicular to the direction of propagation ( z ).  The next task is to find the magnetic field that couples to this electric field solution.  Suppose the coordinate axis are picked so that

\begin{equation}\label{eqn:uwavesLecture2:360}
\BE(z, t) = \BE_0 E_x( z \pm v_\phi t).
\end{equation}

Using Faraday's law \( \spacegrad \cross \BE = -\mu \PDi{t}{\BH} \) gives
\begin{dmath}\label{eqn:uwavesLecture2:380}
\spacegrad \cross (\BE_0 E_x)
=
(\spacegrad \cross \BE_0) E_x
-\BE_0 \cross \spacegrad E_x
=
-\BE_0 \cross \spacegrad E_x
=
-\BE_0 \cross \zcap \PD{z}{E_x},
\end{dmath}

so

\begin{dmath}\label{eqn:uwavesLecture2:460}
-\mu \PD{t}{\BH} =
\lr{ \zcap \cross \BE_0 } \PD{z}{E_x}.
\end{dmath}

Because of the \( E_x( z \pm v_\phi t ) \) dependence on \( z, t \), we must have

\begin{dmath}\label{eqn:uwavesLecture2:480}
\PD{z}{E_x} = \pm \inv{v_\phi} \PD{t}{E_x},
\end{dmath}

or
\begin{dmath}\label{eqn:uwavesLecture2:500}
-\mu \PD{t}{\BH} =
\lr{ \zcap \cross \BE_0 }
= \pm \inv{v_\phi} \PD{t}{E_x}.
\end{dmath}

Integrating with respect to \( t \) and assuming a zero integration constant (physical justification for that?), we have

\begin{dmath}\label{eqn:uwavesLecture2:520}
\BH = \mp E_x(z, t) \frac{\zcap \cross \BE_0}{v_\phi \mu}.
\end{dmath}

The \( v_\phi \mu \) product is the intrinsic wave impedance of the medium

\begin{dmath}\label{eqn:uwavesLecture2:540}
v_\phi \mu = \frac{\mu}{\sqrt{\epsilon\mu}} = \sqrt{\frac{\mu}{\epsilon}} = \eta.
\end{dmath}

The magnetic field can be written in terms of the propagation direction \( \ncap = \mp \zcap \) has

\begin{equation}\label{eqn:uwavesLecture2:560}
\BH
= \mp \frac{\zcap \cross \BE}{\eta}
= \frac{\ncap \cross \BE}{\eta}.
\end{equation}

Note that in free space we have \( \eta \approx 377 \Omega \).

\paragraph{summary}

For 1D wave propagation along the z axis we have

\begin{equation}\label{eqn:uwavesLecture2:580}
\begin{aligned}
\spacegrad^2 \BE &= \epsilon \mu \PDSq{t}{\BE} \\
\spacegrad^2 \BH &= \epsilon \mu \PDSq{t}{\BH} \\
\end{aligned}
\end{equation}

where

\begin{equation}\label{eqn:uwavesLecture2:600}
\begin{aligned}
\BE &= \BE_0 f( z \mp v_\phi t ) \\
0 &= \BE_0 \cdot \zcap \\
\BH &= \pm \frac{\zcap \cross \BE_0}{\eta} f( z \mp v_\phi t )
\end{aligned}
\end{equation}

Note that \( \pm \zcap \) is the direction of propagation, with \( +z \) being the forward wave, and \( -z \) the backwards wave.

The phase velocity is

\begin{equation}\label{eqn:uwavesLecture2:620}
v_\phi = \frac{1}{\sqrt{\mu\epsilon}},
\end{equation}

and the intrinsic wave impedance is

\begin{equation}\label{eqn:uwavesLecture2:640}
\eta = \sqrt{\frac{\mu}{\epsilon}}.
\end{equation}

Notes:

\begin{itemize}
\item \( \BE, \BH \) are perpendicular to the direction of propagation \( \zcap \).  These are transverse waves.
\item \( \Abs{\BE}/\Abs{\BH} = \eta \).  An analogy with transmission lines is
\begin{equation}\label{eqn:uwavesLecture2:660}
\begin{aligned}
V &\leftrightarrow E \\
I &\leftrightarrow H \\
Z_0 &\leftrightarrow \eta
\end{aligned}
\end{equation}
\item \( \BE, \BH \) are orthogonal to each other as sketched in \cref{fig:deck2Waves:deck2WavesFig1}.

\imageFigure{../figures/ece1236-microwaves/deck2WavesFig1}{Transverse electric and magnetic propagation.}{fig:deck2Waves:deck2WavesFig1}{0.2}
\end{itemize}

\section{Time harmonic fields}
\index{time harmonic fields}

Assume that the fields vary sinusoidally and use phasor notation

\begin{equation}\label{eqn:uwavesLecture2:680}
\BE = \Real\lr{ \BE(\Br) e^{j \omega t} },
\end{equation}

where \( \BE(\Br) \) is a (complex) vector-valued phasor.  Notes:

\begin{enumerate}[(i)]
\item Physicists use a \( e^{-j \omega t} \) notation for the time dependence.
\item Sometimes RMS values are used instead of peak values, so that
\begin{equation}\label{eqn:uwavesLecture2:700}
\BE = \sqrt{2} \Real\lr{ \BE(\Br) e^{j \omega t} },
\end{equation}

and \( \BE(\Br) \) is an RMS phasor.
\item The time derivative \( \PDi{t}{} \rightarrow j\omega \) in Maxwell's equations.
\item The \( e^{j \omega t} \) variation is dropped for convenience.
\end{enumerate}

The \( e^{j \omega t} \) time harmonic dependence is not restrictive because one can synthesize any function of t using a Fourier transform

\begin{equation}\label{eqn:uwavesLecture2:720}
\calF \lr{ f(t) } = \tilde{f}(\omega) = \int_{-\infty}^\infty f(t) e^{-j \omega t} dt,
\end{equation}

where \( \tilde{f}(\omega) \) is the spectrum.  If the spectrum is known then the inverse Fourier transform is

\begin{equation}\label{eqn:uwavesLecture2:740}
f(t) = \inv{2 \pi} \int_{-\infty}^\infty \tilde{f}(\omega) e^{j \omega t} dt,
\end{equation}

\index{time harmonic}
If the response to a time harmonic signal \( e^{j \omega t} \) is known then the response to an arbitrary waveform \( f(t) \) can be synthesized, as shown in
\cref{fig:deck2Waves:deck2WavesFig2}
where the waveform input is \( f(t) = \inv{2 \pi} \int_{-\infty}^\infty \tilde{f}(\omega) e^{j \omega t} dt \) and the waveform output is \( \inv{2 \pi} \int_{-\infty}^\infty \tilde{f}(\omega) G(\omega) e^{j \omega t} dt \)

\imageFigure{../figures/ece1236-microwaves/deck2WavesFig2}{System frequency response.}{fig:deck2Waves:deck2WavesFig2}{0.2}

If the fields are known in phasor form \( \BE(\Br, \omega ) \) then the response to an arbitrary signal with spectrum \( \tilde{f}(\omega) \) is

\begin{equation}\label{eqn:uwavesLecture2:760}
\BE(\Br, t ) = \inv{2 \pi} \int_{-\infty}^\infty \BE(\Br, \omega) \tilde{f}(\omega) e^{j \omega t} dt.
\end{equation}

FIXME: unsure what \( \tilde{f}(\omega) \) is here?  Isn't \( \BE(\Br, \omega) \) the spectrum of the field?  Is this supposed to be the field response to the system \( G(\omega) \) ?

\paragraph{Maxwell's equations in phasor form}
\index{phasor}

Faraday's law, the Ampere-Maxwell law, the continuity equation, and the (linear, homogeneous, isotropic) constitutive relations respectively are
\begin{subequations}
\label{eqn:uwavesLecture2:780}
\begin{equation}\label{eqn:uwavesLecture2:800}
\spacegrad \cross \BE = - j \omega \BB
\end{equation}
\begin{equation}\label{eqn:uwavesLecture2:820}
\spacegrad \cross \BH = \BJ + j \omega \BD
\end{equation}
\begin{equation}\label{eqn:uwavesLecture2:840}
\spacegrad \cdot \BJ = - j \omega \rho
\end{equation}
\begin{equation}\label{eqn:uwavesLecture2:860}
\begin{aligned}
\BB &= \mu \BH \\
\BD &= \epsilon \BE \\
\end{aligned}
\end{equation}
\end{subequations}

Note that \( \BD(\Br, \omega) = \epsilon(\omega) \BE(\Br, \omega) \) becomes a convolution in the time domain

\begin{equation}\label{eqn:uwavesLecture2:880}
\BD(\Br, t) = \int \BE(\Br, t - \tau) \epsilon(\tau) d\tau.
\end{equation}

This describes the delay between the \( \BE \) field and the \( \BD \) field, the polarization.

\paragraph{1D propagation, and sinusoidal waves}

In a source free, loss less ( \( \sigma = 0 \) ) simple medium the electric field satisfies

\begin{equation}\label{eqn:uwavesLecture2:900}
\spacegrad^2 \BE - \epsilon \mu \PDSq{t}{\BE} = 0
\end{equation}

In a phasor form the field is

\begin{equation}\label{eqn:uwavesLecture2:920}
\BE(\Br, t) = \Real\lr{ \BE(\Br) e^{j \omega t} },
\end{equation}

so the wave equation becomes

\begin{equation}\label{eqn:uwavesLecture2:940}
\spacegrad^2 \BE - \epsilon \mu (j\omega)^2 \BE = 0,
\end{equation}

or
\boxedEquation{eqn:uwavesLecture2:960}{
\spacegrad^2 \BE + \beta^2 \BE = 0.
}

where the propagation constant, or wave number is

\begin{equation}\label{eqn:uwavesLecture2:980}
\beta = \omega \sqrt{\epsilon \mu} = \frac{\omega}{v_\phi} = \frac{2 \pi}{\lambda},
\end{equation}

where \( \omega = 2 \pi f \) is the angular frequency and \( T = 1/f \) is the period.  For 1D propagation along the z-axis the fields are

\begin{equation}\label{eqn:uwavesLecture2:1000}
\begin{aligned}
E_x(z, t) &= E_0^{+} \cos(\omega t - \beta z) \\
H_x(z, t) &= \frac{E_0^{+}}{\eta} \cos(\omega t - \beta z).
\end{aligned}
\end{equation}

These correspond to phasors
\begin{equation}\label{eqn:uwavesLecture2:1020}
\begin{aligned}
E_x(z) &= E_0^{+} e^{-j k z } \\
H_x(z) &= \frac{E_0^{+}}{\eta} e^{-j k z }.
\end{aligned}
\end{equation}

The peak velocity of either the electric or magnetic fields travels with the phase velocity as sketched in \cref{fig:deck2Waves:deck2WavesFig3}.

\imageFigure{../figures/ece1236-microwaves/deck2WavesFig3}{Phase velocity propagation.}{fig:deck2Waves:deck2WavesFig3}{0.3}

\paragraph{Plane waves}
\index{plane wave}

Consider

\begin{equation}\label{eqn:uwavesLecture2:1040}
E_x(z, t) = E_0^{+} \cos(\omega t - \beta z),
\end{equation}

which had the phasor form \( E_x = E_0^{+} e^{-j \beta z} \).  The phase of the wave is given by

\begin{equation}\label{eqn:uwavesLecture2:1060}
\Phi = \omega t - \beta z.
\end{equation}

At a fixed time \( t \), the locus of the points with the same phase is given by the equation

\begin{equation}\label{eqn:uwavesLecture2:1080}
\beta z = \omega t - \Phi = \textrm{constant}.
\end{equation}

\( \beta z = \textrm{constant} \) describes planes perpendicular to the \( z \) axis, as sketched in \cref{fig:deck2Waves:deck2WavesFig4}.

\imageFigure{../figures/ece1236-microwaves/deck2WavesFig4}{Equiphase surfaces.}{fig:deck2Waves:deck2WavesFig4}{0.2}

Notes

\begin{itemize}
\item Since the equiphase surfaces are planes, we deal with plane waves.
\item \( \beta( z + \lambda ) = \beta z + 2 \pi \).  i.e. wavefronts are separated in space by one wavelength \( \lambda \).
\item velocity of wavefront \( \beta dz = \omega dt \) means that phase velocity is

\begin{equation}\label{eqn:uwavesLecture2:1100}
\frac{dz}{dt} = \frac{\omega}{\beta} = \inv{\sqrt{\mu \epsilon} },
\end{equation}

or
\begin{dmath}\label{eqn:uwavesDeck2WaveEquationCore:1500}
v_\phi = \frac{\omega}{\beta}.
\end{dmath}

Since \( \beta z - \omega t = \beta ( z - v_\phi t ) = \textrm{constant} \) we also have

\begin{dmath}\label{eqn:uwavesDeck2WaveEquationCore:1520}
\frac{dz}{dt} - v_\phi = 0,
\end{dmath}

or

\begin{dmath}\label{eqn:uwavesDeck2WaveEquationCore:1540}
v_\phi = \frac{\lambda}{T}.
\end{dmath}

These can be combined to give
\begin{dmath}\label{eqn:uwavesDeck2WaveEquationCore:1560}
\beta
= \frac{\omega}{v_\phi}
= \frac{\omega T}{\lambda}
= \frac{2 \pi f T}{\lambda}
= \frac{2 \pi}{\lambda}.
\end{dmath}

\end{itemize}

\section{Poynting Theorem}

\paragraph{Poynting vector}
\index{Poynting theorem}
\index{Poynting vector}

Consider a volume V, surrounded by a closed surface S as sketched in \cref{fig:deck2Waves:deck2WavesFig5}.

\imageFigure{../figures/ece1236-microwaves/deck2WavesFig5}{Poynting volume.}{fig:deck2Waves:deck2WavesFig5}{0.2}

\index{instantaneous power density}
The instantaneous power density outflowing the surfaces S s given by the Poynting vector

\begin{equation}\label{eqn:uwavesLecture2:1120}
\BP = \bcE \cross \bcH, \qquad (\si{W/m^2}),
\end{equation}

where \( \bcE \) and \( \bcH \) are time-dependent fields.  For time harmonic fields

\begin{equation}\label{eqn:uwavesLecture2:1140}
\begin{aligned}
\bcE &= \Real \lr{ \BE e^{j \omega t} } \\
\bcH &= \Real \lr{ \BH e^{j \omega t} }
\end{aligned}
\end{equation}

Hence

\begin{dmath}\label{eqn:uwavesLecture2:1160}
\BP
= \bcE \cross \bcH
= \inv{4}
\lr{ \BE e^{j \omega t} +  \BE^\conj e^{-j \omega t} }
\cross
\lr{ \BH e^{j \omega t} +  \BH^\conj e^{-j \omega t} }
=
\inv{4}
\lr{
(\BE \cross \BH) e^{2 j \omega t}
+ (\BE^\conj \cross \BH)
+ (\BE \cross \BH^\conj)
(\BE^\conj \cross \BH^\conj) e^{-2 j \omega t}
}.
\end{dmath}

Therefore the time-average Poynting vector is

\boxedEquation{eqn:uwavesLecture2:1180}{
\BS = \inv{T} \int_0^T \BP dt
= \inv{2} \Real\lr{ \BE \cross \BH^\conj } \qquad (\si{W/m^2}).
}

The total time-average power outflowing S is

\begin{equation}\label{eqn:uwavesLecture2:1200}
P = \inv{2} \oint \Real\lr{ \BE \cross \BH^\conj } \cdot d\BS.
\end{equation}

\paragraph{Poynting theorem}

Consider again a volume V enclosed in a surface S.  Assume that electric currents may exist in V and that V is filled with a material characterized by \( \epsilon, \mu \) and a conductivity \( \sigma \), where \( \epsilon = \epsilon' -j \epsilon'' \), and \( \mu = \mu' - j \mu'' \) as illustrate in \cref{fig:deck2Waves:deck2WavesFig6}.

\imageFigure{../figures/ece1236-microwaves/deck2WavesFig6}{Poynting volume with sources.}{fig:deck2Waves:deck2WavesFig6}{0.2}

From Faraday's law \( \spacegrad \cross \BE = - j \omega \mu \BH \), and Ampere's law
\( \spacegrad \cross \BH = \BJ + j \omega \epsilon \BE \), we have

\begin{equation}\label{eqn:uwavesLecture2:1220}
\begin{aligned}
\BH^\conj \cdot \lr{ \spacegrad \cross \BE } &= - j \omega \mu \Abs{\BH}^2 \\
\BE \cdot \lr{ \spacegrad \cross \BH^\conj } &= \BJ^\conj \cdot \BE - j \omega \epsilon^\conj \Abs{\BE}^2
\end{aligned}
\end{equation}

We assume that the current is a sum of sources and induced currents \( \BJ = \BJ_\txts + \BJ_{\mathrm{ind}} \), where \( \BJ_{\mathrm{ind}} = \sigma \BE \).  Recall that

\begin{equation}\label{eqn:uwavesLecture2:1240}
\spacegrad \cdot \lr{ \BE \cross \BH^\conj } =
\BH^\conj \cdot \lr{ \spacegrad \cross \BE }
-\BE^\conj \cdot \lr{ \spacegrad \cross \BH^\conj }.
\end{equation}

Hence

\begin{equation}\label{eqn:uwavesLecture2:1260}
\spacegrad \cdot \lr{ \BE \cross \BH^\conj }
=
- j \omega \mu \Abs{\BH}^2 - \BJ_\txts^\conj \cdot \BE - \sigma \Abs{\BE}^2 + j \omega \epsilon^\conj \Abs{\BE}^2.
\end{equation}

Now integrate over V, using the divergence theorem

\begin{dmath}\label{eqn:uwavesLecture2:1280}
\int_V
\spacegrad \cdot \lr{ \BE \cross \BH^\conj } dV
=
\oint \lr{ \BE \cross \BH^\conj } \cdot d\BS
=
- \sigma \int_V \Abs{\BE}^2 dV
+ j \omega \int_V \lr{ \epsilon^\conj \Abs{\BE}^2 -\mu \Abs{\BH}^2 } dV
- \int_V \BJ_\txts^\conj \cdot \BE dV,
\end{dmath}

or
\begin{dmath}\label{eqn:uwavesLecture2:1300}
-\inv{2} \int_V \BJ_\txts^\conj \cdot \BE dV,
=
+ \inv{2} \oint \lr{ \BE \cross \BH^\conj } \cdot d\BS
+ \frac{\sigma}{2} \int_V \Abs{\BE}^2 dV
- j \frac{\omega}{2} \int_V \lr{ \epsilon^\conj \Abs{\BE}^2 -\mu \Abs{\BH}^2 } dV
=
+ \inv{2} \oint \lr{ \BE \cross \BH^\conj } \cdot d\BS
+ \frac{\sigma}{2} \int_V \Abs{\BE}^2 dV
+ \frac{\omega}{2} \int_V \lr{ \epsilon'' \Abs{\BE}^2 + \mu'' \Abs{\BH}^2 } dV
+ j \frac{\omega}{2} \int_V \lr{ -\epsilon' \Abs{\BE}^2 + \mu' \Abs{\BH}^2 } dV
\end{dmath}

This is the mathematical manifestation of Poynting's theorem which is nothing else but the conservation of power with volume V.  We can now identify the following terms:

\begin{itemize}
\item Power generated by current sources
\begin{equation}\label{eqn:uwavesLecture2:1320}
P_\txts = -\inv{2} \int_V \BJ_\txts^\conj \cdot \BE dV
\end{equation}
\item Complex power flowing out of V
\begin{equation}\label{eqn:uwavesLecture2:1340}
P_\txto = \inv{2} \oint \lr{ \BE \cross \BH^\conj } \cdot d\BS
\end{equation}
\item Power loss due to materials
\begin{equation}\label{eqn:uwavesLecture2:1360}
P_\txtm = \frac{\omega}{2} \int_V \lr{ \epsilon'' \Abs{\BE}^2 + \mu'' \Abs{\BH}^2 } dV
\end{equation}
\item Ohmic losses
\begin{equation}\label{eqn:uwavesLecture2:1380}
P_\txtc = \frac{\sigma}{2} \int_V \Abs{\BE}^2 dV
\end{equation}
\item Time average stored electric energy
\begin{equation}\label{eqn:uwavesLecture2:1400}
W_\txte = \inv{4} \epsilon' \int_V \Abs{\BE}^2 dV
\end{equation}
\item Time average stored magnetic energy
\begin{equation}\label{eqn:uwavesLecture2:1420}
W_\txte = \inv{4} \mu' \int_V \Abs{\BH}^2 dV
\end{equation}
\end{itemize}

The Poynting theorem is a complex power balance of the form
\begin{equation}\label{eqn:uwavesLecture2:1440}
P_\txts = P_\txto + P_\txtm + P_\txtc + 2 j \omega \lr{ W_\txtm - W_\txte }
\end{equation}

The real part of the above equation is related to the time average power.  i.e.

\begin{equation}\label{eqn:uwavesLecture2:1460}
\Real(P_\txts) = \Real(P_\txto) + P_\txtm + P_\txtc.
\end{equation}

On the other hand the imaginary part is related to the net stored energy with V.  i.e.

\begin{equation}\label{eqn:uwavesLecture2:1480}
\Imag( P_\txts ) = 2 \omega \lr{ W_\txtm - W_\txte } + \Imag( P_\txto ).
\end{equation}

We can now use Poynting theorem to calculate losses in waveguides.
