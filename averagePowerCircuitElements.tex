%
% Copyright � 2016 Peeter Joot.  All Rights Reserved.
% Licenced as described in the file LICENSE under the root directory of this GIT repository.
%
%{
%\input{../blogpost.tex}
%\renewcommand{\basename}{averagePowerCircuitElements}
%\renewcommand{\dirname}{notes/phy1520/}
%%\newcommand{\dateintitle}{}
%%\newcommand{\keywords}{}
%
%\input{../peeter_prologue_print2.tex}
%
%\usepackage{peeters_layout_exercise}
%\usepackage{peeters_braket}
%\usepackage{peeters_figures}
%
%\beginArtNoToc
%
%\generatetitle{Average power for circuit elements}
\section{Average power for circuit elements}
%\label{chap:averagePowerCircuitElements}
\index{average power}

In \citep{pozar2009microwave} \S 2.2 is a comparison of field energy expressions with their circuit equivalents.  It's clearly been too long since I've worked with circuits, because I'd forgotten all the circuit energy expressions:
\begin{equation}\label{eqn:averagePowerCircuitElements:20}
\begin{aligned}
W_\txtR &= \frac{R}{2} \Abs{I}^2 \\
W_\txtC &= \frac{C}{4} \Abs{V}^2 \\
W_\txtL &= \frac{L}{4} \Abs{I}^2 \\
W_\txtG &= \frac{G}{2} \Abs{V}^2 \\
\end{aligned}
\end{equation}

Here's a recap of where these come from
\paragraph{Energy lost to resistance}
\index{resistance loss}
Given
\begin{equation}\label{eqn:averagePowerCircuitElements:40}
v(t) = R i(t),
\end{equation}
the average power lost to a resistor is
\begin{equation}\label{eqn:averagePowerCircuitElements:60}
\begin{aligned}
p_\txtR
&= \inv{T} \int_0^T v(t) i(t) dt
\\ &= \inv{T} \int_0^T \Real( V e^{j \omega t} ) \Real( I e^{j \omega t} ) dt
\\ &= \inv{4 T} \int_0^T
\lr{V e^{j \omega t} + V^\conj e^{-j \omega t} }
\lr{I e^{j \omega t} + I^\conj e^{-j \omega t} }
dt
\\ &= \inv{4 T} \int_0^T
\lr{
V I e^{2 j \omega t} + V^\conj I^\conj e^{-2 j \omega t}
+ V I^\conj + V^\conj I
}
dt
\\ &= \inv{2} \Real( V I^\conj )
\\ &= \inv{2} \Real( I R I^\conj )
\\ &= \frac{R}{2} \Abs{I}^2.
\end{aligned}
\end{equation}

Here it is assumed that the averaging is done over some integer multiple of the period, which kills off all the exponentials.
\paragraph{Energy stored in a capacitor}
\index{energy!capacitor}
I tried the same sort of analysis for a capacitor in phasor form, but everything canceled out.  Referring to \citep{irwin2007bec}, the approach used to figure this out is to operate first strictly in the time domain.  Specifically, for the capacitor where \( i = C dv/dt \) the power supplied up to a time \( t \) is
\begin{equation}\label{eqn:averagePowerCircuitElements:80}
\begin{aligned}
p_\txtC(t)
&= \int_{-\infty}^t C \frac{dv}{dt} v(t) dt
\\ &= \inv{2} C v^2(t).
\end{aligned}
\end{equation}

The \( v^2(t) \) term can now be expanded in terms of phasors and averaged for
\begin{equation}\label{eqn:averagePowerCircuitElements:100}
\begin{aligned}
\overbar{p}_\txtC
&= \frac{C}{2T} \int_0^T \inv{4}
\lr{ V e^{j \omega t} + V^\conj e^{-j \omega t} }
\lr{ V e^{j \omega t} + V^\conj e^{-j \omega t} } dt
\\ &= \frac{C}{2T} \int_0^T \inv{4}
2 \Abs{V}^2 dt
\\ &= \frac{C}{4} \Abs{V}^2.
\end{aligned}
\end{equation}
\paragraph{Energy stored in an inductor}
\index{energy!inductor}
The inductor energy is found the same way, with
\begin{equation}\label{eqn:averagePowerCircuitElements:120}
\begin{aligned}
p_\txtL(t)
&= \int_{-\infty}^t L \frac{di}{dt} i(t) dt
\\ &= \inv{2} L i^2(t),
\end{aligned}
\end{equation}
which leads to
\begin{equation}\label{eqn:averagePowerCircuitElements:140}
\overbar{p}_\txtL
= \frac{L}{4} \Abs{I}^2.
\end{equation}
\paragraph{Energy lost due to conductance}
\index{energy!conductance}
Finally, we have conductance.  In phasor space that is defined by
\begin{equation}\label{eqn:averagePowerCircuitElements:160}
G = \frac{I}{V} = \inv{R},
\end{equation}
so power lost due to conductance follows from power lost due to resistance.  In the average we have
\begin{equation}\label{eqn:averagePowerCircuitElements:180}
\begin{aligned}
p_\txtG
&= \inv{2 G} \Abs{I}^2
\\ &= \inv{2 G} \Abs{V G}^2
\\ &= \frac{G}{2} \Abs{V}^2
\end{aligned}
\end{equation}
%}
%\EndArticle
