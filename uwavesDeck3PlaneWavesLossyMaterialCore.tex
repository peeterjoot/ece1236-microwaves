%
% Copyright © 2016 Peeter Joot.  All Rights Reserved.
% Licenced as described in the file LICENSE under the root directory of this GIT repository.
%
\section{Lossy media}
\index{lossy media}

In a lossy medium characterized by \( \gamma = \alpha + j \beta \) such that the wave behaves like \( e^{-\gamma z } \) for propagation along \( +z \), or

\begin{equation}\label{eqn:uwavesLecture3:20}
e^{-\gamma z} = e^{-\alpha z} e^{-j \beta z}
\end{equation}

\index{attenuation}
The \( e^{-\alpha z} \) term introduces an exponential attenuation with \( z \).  The wave equation in the phasor domain is then

\begin{equation}\label{eqn:uwavesLecture3:40}
\spacegrad^2 \BE - \gamma^2 \BE = 0,
\end{equation}

where

\begin{equation}\label{eqn:uwavesLecture3:60}
\gamma^2 = - \omega^2 \mu \epsilon_{\mathrm{eff}}.
\end{equation}

\index{permittivity!complex}
The effective complex permittivity \( \epsilon_{\mathrm{eff}} = \epsilon' -j \epsilon'' \) is obtained from Ampere's law

\begin{dmath}\label{eqn:uwavesLecture3:80}
\spacegrad \cross \BH
= \BJ + j \omega \epsilon \BE
= \sigma \BE + j \omega \epsilon \BE
= \lr{ \sigma + j \omega \epsilon } \BE
= j \omega \lr{ \epsilon -j \frac{\sigma}{\omega} } \BE ,
\end{dmath}

so
\begin{equation}\label{eqn:uwavesLecture3:100}
\begin{aligned}
\epsilon_{\mathrm{eff}} &= \epsilon' -j \epsilon'' \\
\epsilon' &= \epsilon \\
\epsilon'' &= \frac{\sigma}{\omega}
\end{aligned}
\end{equation}

Since \( \gamma = j \omega \sqrt{ \mu \epsilon_{\mathrm{eff}} } \) and \( \gamma^2 = -\omega^2 \mu \epsilon_{\mathrm{eff}} \) we see that

\begin{dmath}\label{eqn:uwavesLecture3:120}
(\alpha + j \beta)^2
= (\alpha^2 - \beta^2) + j 2 \alpha \beta
= - \omega^2 \mu \epsilon_{\mathrm{eff}}
= - \omega^2 \mu \epsilon'
  + j \omega^2 \mu \epsilon'',
\end{dmath}

therefore

\begin{equation}\label{eqn:uwavesLecture3:140}
\begin{aligned}
\alpha^2 - \beta^2 &= - \omega^2 \mu \epsilon' \\
2 \alpha \beta &= \omega^2 \mu \epsilon''.
\end{aligned}
\end{equation}

Solving for \( \alpha \) and \( \beta \) yields

\begin{equation}\label{eqn:uwavesLecture3:160}
\begin{aligned}
\alpha &= \omega \sqrt{ \frac{\mu \epsilon'}{2} \lr{ \sqrt{ 1 + \lr{\frac{\epsilon''}{\epsilon'}}^2 } - 1 } }, \qquad \si{Np/m}  \\
\beta &= \omega \sqrt{ \frac{\mu \epsilon'}{2} \lr{ \sqrt{ 1 + \lr{\frac{\epsilon''}{\epsilon'}}^2 } + 1 } } , \qquad \si{rad/m} \\
\end{aligned}
\end{equation}

Assuming propagation along \( + \zcap \)

\begin{dmath}\label{eqn:uwavesLecture3:180}
\BE(z)
= \xcap E_x(z)
= \xcap E_{x0} e^{-\gamma z}
= \xcap E_{x0} e^{-\alpha z} e^{-j \beta z}
\end{dmath}

The magnetic field can be determined from

\begin{dmath}\label{eqn:uwavesLecture3:200}
\spacegrad \cross \BE = - j \omega \mu \BH,
\end{dmath}

or

\begin{dmath}\label{eqn:uwavesLecture3:220}
\BH
= \ycap \frac{E_x}{\eta_\txtc}
= \ycap \frac{E_{x0}}{\eta_\txtc} e^{-\alpha z} e^{-j \beta z},
\end{dmath}

where
\begin{dmath}\label{eqn:uwavesLecture3:240}
\eta_\txtc
= \sqrt{\frac{\mu}{\epsilon_{\mathrm{eff}}}}
= \sqrt{\frac{\mu}{\epsilon'}} \lr{ 1 - j \frac{\epsilon''}{\epsilon'} }^{-1/2} \qquad \Omega,
\end{dmath}

is the complex intrinsic impedance of the medium.  Note that since \( \eta_\txtc \) is complex the electric and magnetic fields are no longer in phase.

\section{Skin depth}
\index{skin depth}

Note that

\begin{equation}\label{eqn:uwavesLecture3:260}
\begin{aligned}
\Abs{ E_x } &= \Abs{ E_{x0} } e^{-\alpha z} \\
\Abs{ H_y } &= \Abs{ H_{x0} } e^{-\alpha z}.
\end{aligned}
\end{equation}

The skin depth \( \delta_\txts \) is defined as the distance that the wave needs to travel to reduce its magnitude by \( 1/e \).  Hence

\begin{equation}\label{eqn:uwavesLecture3:280}
e^{ -\alpha \delta_\txts } = e^{-1},
\end{equation}

or

\boxedEquation{eqn:uwavesLecture3:300}{
\delta_\txts  = \frac{1}{\alpha}.
}

This is sketched in \cref{fig:deck3Lossy:deck3LossyFig1}.

\imageFigure{../figures/ece1236-microwaves/deck3LossyFig1}{Skin depth.}{fig:deck3Lossy:deck3LossyFig1}{0.2}

\paragraph{Low loss dielectrics}
\index{dielectric!low loss}

A low loss dielectric medium is characterized by \( \epsilon'' \ll \epsilon' \), so \( \epsilon \approx \epsilon' \).  This gives

\begin{dmath}\label{eqn:uwavesLecture3:320}
\gamma
=
j \omega \sqrt{\mu \epsilon' } \lr{ 1 -j \frac{\epsilon''}{\epsilon'}}^{1/2}
\approx
j \omega \sqrt{\mu \epsilon' } \lr{ 1 - j \frac{\epsilon''}{2 \epsilon'}}.
\end{dmath}

Hence

\begin{dmath}\label{eqn:uwavesLecture3:340}
\alpha
\approx
\omega \sqrt{\mu \epsilon'} \frac{\epsilon''}{2 \epsilon'}
=
\frac{\sigma}{2} \sqrt{\frac{\mu}{\epsilon'}}
\approx
\frac{\sigma}{2} \sqrt{\frac{\mu}{\epsilon}}.
\end{dmath}

\begin{dmath}\label{eqn:uwavesLecture3:360}
\beta
\approx
\omega \sqrt{\mu \epsilon'}
\approx
\omega \sqrt{\mu \epsilon},
\end{dmath}

which is the same as in the lossless medium \( \epsilon \).  Also in this case

\begin{dmath}\label{eqn:uwavesLecture3:380}
\eta_\txtc
=
\sqrt{\frac{\mu}{\epsilon'}} \lr{ 1 - j \frac{\epsilon''}{\epsilon'} }^{-1/2}
\approx
\sqrt{\frac{\mu}{\epsilon'}} \lr{ 1 + j \frac{\epsilon''}{2 \epsilon'} }
=
\sqrt{\frac{\mu}{\epsilon'}} \lr{ 1 + j \frac{\sigma}{2 \omega \epsilon'} }
\approx
\sqrt{\frac{\mu}{\epsilon'}},
\end{dmath}

(for \( \sigma \ll \omega \epsilon \), (a good dielectric), so

\begin{dmath}\label{eqn:uwavesLecture3:400}
\eta_\txtc \approx \sqrt{\frac{\mu}{\epsilon}},
\end{dmath}

which is the same as if the medium is lossless.

\paragraph{Good conductors}
\index{conductor!good}

In this case \( \sigma \gg \omega \epsilon \) so that

\begin{dmath}\label{eqn:uwavesLecture3:420}
\gamma
=
j \omega \sqrt{ \epsilon_{\mathrm{eff}} \mu }
=
j \omega \sqrt{ \epsilon \mu } \lr{ 1 - j \frac{\sigma}{\omega \epsilon} }^{1/2}
\approx
j \omega \sqrt{ \epsilon \mu } \sqrt{- j \frac{\sigma}{\omega \epsilon} }
=
j \omega \sqrt{ \epsilon \mu } \sqrt{\frac{\sigma}{\omega \epsilon} } e^{-j \pi/4}
=
j \omega \sqrt{ \frac{ \mu \sigma}{2 \omega} } ( 1 - j )
=
\sqrt{ \frac{ \mu \sigma \omega}{2 } } ( 1 + j )
= \alpha + j \beta,
\end{dmath}

Hence

\begin{equation}\label{eqn:uwavesLecture3:440}
\alpha \approx \beta \approx \sqrt{\frac{\omega \mu \sigma}{2}} = \sqrt{ \pi f \mu \sigma }.
\end{equation}

Since \( \spacegrad \cross \BE = - \mu (j \omega) \BH \), and with a \( \BE \propto e^{-\gamma z } \) dependence, we have

\begin{dmath}\label{eqn:uwavesLecture3:480}
\eta_\txtc = \frac{\Abs{\BE}}{\Abs{\BH}} = \frac{- \mu (j \omega)}{-\gamma} = \frac{j \omega \mu}{\gamma}
\end{dmath}

so
\begin{dmath}\label{eqn:uwavesLecture3:460}
\eta_\txtc
=
\frac{j \omega \mu}{ \gamma}
=
\frac{j \omega \mu}{
\sqrt{ \frac{ \mu \sigma \omega}{2 } } ( 1 + j )
}
=
\frac{j (1 - j ) \omega \mu}{
2 \sqrt{ \frac{ \mu \sigma \omega}{2 } }
}
=
(j + 1) \sqrt{ \frac{ \omega \mu }{ 2 \sigma} }
=
(j + 1) \frac{\alpha}{\sigma}.
\end{dmath}

The skin depth for a good conductor is

\begin{dmath}\label{eqn:uwavesLecture3:500}
\delta_\txts = \inv{\alpha} = \sqrt{\frac{2}{ \omega \mu \sigma } } = \inv{\sqrt{\pi f \mu \sigma}}.
\end{dmath}

\makeexample{Copper skin depth.}{example:uwavesLecture3:1}{
Copper has a conductivity of \( \sigma = 5.8 \times 10^7 \si{S/m} \).  What is the skin depth
\( \delta_\txts = \ifrac{1}{\sqrt{\pi f \mu_0 \sigma}} \) ?

\begin{itemize}
\item At 1 \si{GHz}, \( \delta_\txts \approx 2.1 \mu \si{m} \)
\item At 3 \si{GHz}, \( \delta_\txts \approx 1.2 \mu \si{m} \)
\item At 10 \si{GHz}, \( \delta_\txts \approx 0.6 \mu \si{m} \)
\end{itemize}

We can conclude that a very thin metallic film can effectively shield RF energy.

} % example

\section{Skin effect}
\index{skin effect}

At DC the current through a wire flows uniformly throughout the entire cross section of a wire as sketched in \cref{fig:deck3Lossy:deck3LossyFig2}.

\imageFigure{../figures/ece1236-microwaves/deck3LossyFig2}{DC current distribution in wire.}{fig:deck3Lossy:deck3LossyFig2}{0.2}

The current occupies the entire cross section \( \pi b^2 \), so the DC resistance is

\begin{dmath}\label{eqn:uwavesLecture3:540}
R_{\textrm{DC}} = \inv{\sigma} \frac{l}{\pi b^2}.
\end{dmath}

However, with RF (like microwave) signals the current only exists effectively within on skin depth, as sketched in \cref{fig:deck3Lossy:deck3LossyFig3}.

\imageFigure{../figures/ece1236-microwaves/deck3LossyFig3}{RF current distribution in wire.}{fig:deck3Lossy:deck3LossyFig3}{0.2}

therefore the RF resistance is

\begin{dmath}\label{eqn:uwavesLecture3:560}
R_{\textrm{RF}} = \inv{\sigma} \frac{l}{2 \pi b \delta_\txts}.
\end{dmath}

Since \( A_{\textrm{RF}} \ll A = \pi b^2 \),

\begin{dmath}\label{eqn:uwavesLecture3:580}
R_{\textrm{RF}} \gg R_{\textrm{DC}}.
\end{dmath}

Wires become very lossy at microwave frequencies.

\makeexample{Skin effect for copper.}{example:uwavesLecture3:2}{

Consider a copper wire with \( \sigma = 5.8 \times 10^7 \si{S/m} \) of radius \( b = 0.5 \si{mm} \) and length \( 10 \si{m} \).  Find the DC and RF resistance at \( 1, 3 \) and \( 10 \si{GHz} \).  The DC resistance is

\begin{dmath}\label{eqn:uwavesLecture3:520}
R_{\textrm{DC}} = \inv{\sigma} \frac{l}{\pi b^2} = \frac{ 10 } { 5.8 \times 10^7 \times \pi (0.0005)^2 } = 0.22 \Omega.
\end{dmath}

On the other hand

\begin{dmath}\label{eqn:uwavesLecture3:600}
R_{\textrm{RF}} = \inv{\sigma} \frac{l}{2 \pi b \delta_s}.
\end{dmath}

So, at

\begin{itemize}
\item \( 1 \si{GHz} \), \( R_{\textrm{RF}} = 26 \Omega \)
\item \( 3 \si{GHz} \), \( R_{\textrm{RF}} = 45.7 \Omega \)
\item \( 10 \si{GHz} \), \( R_{\textrm{RF}} = 91.5 \Omega \)
\end{itemize}

Compare these values with \( R_{\textrm{DC}} = 0.22 \Omega \).

} % example

\makeproblem{Solve for the \( \alpha \), \(\beta\) constants.}{problem:uwavesDeck3PlaneWavesLossyMaterial:1}{
Prove \cref{eqn:uwavesLecture3:160}.

} % problem
