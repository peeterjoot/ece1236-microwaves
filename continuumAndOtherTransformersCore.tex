%
% Copyright © 2016 Peeter Joot.  All Rights Reserved.
% Licenced as described in the file LICENSE under the root directory of this GIT repository.
%
%{
\section{Continuum transformer}
\index{impedance matching}

A non-discrete impedance matching transformation, as sketched in \cref{fig:taperedLines:taperedLinesFig1}, is also possible.

\imageFigure{../../figures/ece1236/taperedLinesFig1}{Tapered impedance matching.}{fig:taperedLines:taperedLinesFig1}{0.2}

\begin{equation}\label{eqn:continuumAndOtherTransformersCore:820}
\Delta \Gamma
= \frac{ (Z + \Delta Z) - Z }{(Z + \Delta Z) + Z}
= \frac{\Delta Z}{2 Z}
\end{equation}

\begin{equation}\label{eqn:continuumAndOtherTransformersCore:840}
\Delta Z \rightarrow 0
\end{equation}

\begin{dmath}\label{eqn:continuumAndOtherTransformersCore:860}
d\Gamma
= \frac{dZ}{2 Z}
= \inv{2} \frac{d (\ln Z)}{dz}
= \frac{Z_0}{Z} \frac{d (Z/Z_0)}{dz}
= \frac{1}{Z} \frac{d Z}{dz}.
\end{dmath}

Hence as we did for multisection transformers, associate \( \Delta \Gamma \) with \( e^{- 2j \beta z} \) as sketched in \cref{fig:taperedLines:taperedLinesFig2}.

\imageFigure{../../figures/ece1236/taperedLinesFig2}{Reflection coefficient over an interval}{fig:taperedLines:taperedLinesFig2}{0.15}

assuming small reflections (i.e. \( Z(z) \) is a slowly varying (adiabatic).  Then

\begin{dmath}\label{eqn:continuumAndOtherTransformersCore:920}
\Gamma(\omega)
= \int_0^L e^{ -2 j \beta z} d\Gamma
= \inv{2}
\int_0^L e^{ -2 j \beta z}  \frac{d (\ln Z)}{dz} dz
\end{dmath}

This supplies the means to calculate the reflection coefficient for any impedance curve.  As with the step impedance matching process, it is assumed that only first order reflections are of interest.

\section{Exponential taper}
\index{exponential taper}

Let
\begin{equation}\label{eqn:continuumAndOtherTransformersCore:620}
Z(z) = Z_0 e^{a z}, \qquad 0 < z < L
\end{equation}

subject to
\begin{equation}\label{eqn:continuumAndOtherTransformersCore:640}
\begin{aligned}
Z(0) &= Z_0 \\
Z(L) &= Z_0 e^{a L} = Z_\txtL,
\end{aligned}
\end{equation}

which gives

\begin{equation}\label{eqn:continuumAndOtherTransformersCore:660}
\ln \frac{Z_\txtL}{Z_0} = a L,
\end{equation}

or
\begin{equation}\label{eqn:continuumAndOtherTransformersCore:680}
a = \inv{L} \ln \frac{Z_\txtL}{Z_0}
\end{equation}

Also

\begin{equation}\label{eqn:continuumAndOtherTransformersCore:700}
\frac{d}{dz} \ln \frac{Z_\txtL}{Z_0}
= \frac{d}{dz} (az)
= a,
\end{equation}

Hence

\begin{dmath}\label{eqn:continuumAndOtherTransformersCore:740}
\Gamma(\omega)
= \inv{2} \int_0^L e^{-2 j \beta z} \frac{d}{dz} \ln \frac{Z_\txtL}{Z_0} dz
= \frac{a}{2}
\int_0^L e^{-2 j \beta z} dz
= \frac{1}{2L} \ln \frac{Z_\txtL}{Z_0}
\evalrange{ \frac{e^{-2 j \beta z} }{ -2 j \beta} }{0}{L}
= \frac{1}{2L \beta} \ln \frac{Z_\txtL}{Z_0}
\frac{ 1 - e^{-2 j \beta L} }{2 j}
= \frac{1}{2} \ln \frac{Z_\txtL}{Z_0} e^{-j \beta L} \frac{\sin( \beta L )}{\beta L},
\end{dmath}

or

\begin{dmath}\label{eqn:continuumAndOtherTransformersCore:940}
\Gamma(\omega) = \frac{1}{2} \ln \frac{Z_\txtL}{Z_0} e^{-j \beta L} \sinc( \beta L ).
\end{dmath}

\begin{enumerate}
\item \( \beta \) is constant with \( Z \) varying: this is good only for TEM lines.
\item \( \Abs{\Gamma} \) decreases with increasing length.
\item An electrical length \( \beta L > \pi \), is required to minimize low frequency mismatch (\( L > \lambda/2\)).
\end{enumerate}

This is sketched in \cref{fig:taperedLines:taperedLinesFig3}.

\imageFigure{../../figures/ece1236/taperedLinesFig3}{Exponential taper reflection coefficient.}{fig:taperedLines:taperedLinesFig3}{0.2}

Want:

\begin{dmath}\label{eqn:continuumAndOtherTransformersCore:760}
\beta L = \pi,
\end{dmath}

or
\begin{dmath}\label{eqn:continuumAndOtherTransformersCore:780}
\frac{\omega_c}{v_\phi} L = \pi
\end{dmath}

where \( \omega_c \) is the cutoff frequency.  This gives

\begin{dmath}\label{eqn:continuumAndOtherTransformersCore:800}
\omega_c = \frac{\pi v_\phi}{L}.
\end{dmath}

\paragraph{Triangular Taper}
\index{triangular taper}

\begin{equation}\label{eqn:continuumAndOtherTransformersCore:960}
Z(z) =
\left\{
\begin{array}{l l}
Z_0 e^{2(z/L)^2 \ln(Z_\txtL/Z_0)} & \quad \mbox{\( 0 \le z \le L/2\)} \\
Z_0 e^{(4z/L - 2 z^2 - 1) \ln(Z_\txtL/Z_0)} & \quad \mbox{\( L/2 \le z \le L\)} \\
\end{array}
\right.
\end{equation}

\begin{equation}\label{eqn:continuumAndOtherTransformersCore:980}
\frac{d}{dz} \ln (Z/Z_0) =
\left\{
\begin{array}{l l}
{(4z/L^2) \ln(Z_\txtL/Z_0)} & \quad \mbox{\( 0 \le z \le L/2\)} \\
{(4/L - 4z/L^2) \ln(Z_\txtL/Z_0)} & \quad \mbox{\( L/2 \le z \le L\)} \\
\end{array}
\right.
\end{equation}

In this case

\begin{dmath}\label{eqn:continuumAndOtherTransformersCore:1000}
\Gamma(\omega) = \frac{1}{2} e^{-\beta L} \ln \frac{Z_\txtL}{Z_0} e^{-j \beta L} \sinc^2( \beta L/2 ).
\end{dmath}

Compared to the exponential taper \( \sinc( \beta L ) \) for the \( \beta L > 2 \pi \) the peaks of \( \Abs{\Gamma} \) are lower, but the first null occurs at \( \beta L = 2 \pi \) whereas for the exponential taper it occurs at \( \beta L = \pi \).  This is sketched in \cref{fig:taperedLines:taperedLinesFig4}.  The price to pay for this is that the zero is at \( 2 \pi \) so we have to make it twice as long to get the ripple down.

\imageFigure{../../figures/ece1236/taperedLinesFig4}{Triangular taper impedance curve.}{fig:taperedLines:taperedLinesFig4}{0.2}

\paragraph{Klopfenstein Taper}
\index{Klopfenstein taper}

For a given taper length \( L \), the Klopfenstein taper is optimum in the sense that the reflection coefficient in the passband is minimum.  Alternatively, for a given minimum reflection coefficient in the passband, the Klopfenstein taper yields the shortest length \( L \).

Definition:

\begin{equation}\label{eqn:continuumAndOtherTransformersCore:1020}
\ln Z = \inv{2} \ln (Z_0 Z_\txtL) + \frac{\Gamma_0}{\cosh A} A^2 \phi(2 z/L -1, A), \qquad 0 \le z \le L,
\end{equation}

where

\begin{equation}\label{eqn:continuumAndOtherTransformersCore:1040}
\phi(x, A) = \int_0^x \frac{I_1(A\sqrt{1 - y^2})}{A \sqrt{1 - y^2}} dy, \qquad \Abs{x} \le 1.
\end{equation}

Here \( I_1(x) \) is the modified Bessel function.  Note that
\begin{equation}\label{eqn:continuumAndOtherTransformersCore:1060}
\begin{aligned}
\phi(0, A) &= 0 \\
\phi(x, 0) &= x/2 \\
\phi(1, A) &= \frac{\cosh A - 1}{A^2}
\end{aligned}
\end{equation}

The resulting reflection coefficient is

\begin{equation}\label{eqn:continuumAndOtherTransformersCore:1080}
\Gamma(\omega)
=
\left\{
\begin{array}{l l}
\Gamma_0 e^{-j \beta L} \frac{\cos\sqrt{(\beta L)^2 - A^2}}{\cosh A} & \quad \mbox{\( \beta L > A \)} \\
\Gamma_0 e^{-j \beta L} \frac{\cos\sqrt{A^2 - (\beta L)^2 }}{\cosh A} & \quad \mbox{\( \beta L < A \)} \\
\end{array}
\right.,
\end{equation}

where as usual

\begin{equation}\label{eqn:continuumAndOtherTransformersCore:1100}
\Gamma_0 = \frac{Z_\txtL - Z_0}{Z_\txtL + Z_0} \approx \inv{2} \ln (Z_\txtL/Z_0).
\end{equation}

The passband is defined by \( \beta L \ge A \) and the maximum ripple in the passband is

\begin{equation}\label{eqn:continuumAndOtherTransformersCore:1120}
\Gamma_m = \frac{\Gamma_0}{\cosh A}.
\end{equation}

\makeexample{Triangular taper vs. exponential taper vs. Klopfenstein taper.}{example:continuumAndOtherTransformersCore:1}{

Design a triangular taper, an exponential taper, and a Klopfenstein taper (with \( \Gamma_m = 0.02 \) ) to match a \( 50 \Omega \) load to a \( 100 \Omega \) line.

\begin{itemize}
\item Triangular taper:

\begin{equation}\label{eqn:continuumAndOtherTransformersCore:1140}
Z(z) =
\left\{
\begin{array}{l l}
Z_0 e^{ 2(z/L)^2 \ln Z_\txtL/Z_0 } & \quad \mbox{\( 0 \le z \le L/2\)} \\
Z_0 e^{ (4 z/L - 2 z^2/L^2 - 1)\ln Z_\txtL/Z_0 } & \quad \mbox{\( L/2 \ge z \ge L \)}
\end{array}
\right.
\end{equation}

The resulting \( \Gamma \) is

\begin{equation}\label{eqn:continuumAndOtherTransformersCore:1160}
\Abs{\Gamma} = \inv{2} \ln (Z_\txtL/Z_0) \sinc^2\lr{ \beta L/2 }.
\end{equation}

\item Exponential taper:

\begin{equation}\label{eqn:continuumAndOtherTransformersCore:1180}
\begin{aligned}
Z(z) &= Z_0 e^{a z}, \qquad 0 \le z \le L \\
a &= \inv{L} \ln (Z_\txtL/Z_0) = \frac{0.693}{L} \\
\Abs{\Gamma} &= \inv{2} \ln (Z_\txtL/Z_0) \sinc( \beta L )
\end{aligned}
\end{equation}

\item Klopfenstein taper:
\index{Klopfenstein taper}
\begin{equation}\label{eqn:continuumAndOtherTransformersCore:1200}
\begin{aligned}
Z(z) &= \inv{2} \ln (Z_\txtL/Z_0) = 0.346 \\
A &= \cosh^{-1}\lr{ \frac{\Gamma_0}{\Gamma_m}} =  \cosh^{-1}\lr{ \frac{0.346}{0.02}} = 3.543 \\
\Abs{\Gamma} &= \Gamma_0 \frac{\cos\sqrt{(\beta L)^2 - A^2}}{\cosh A},
\end{aligned}
\end{equation}

The passband \( \beta L > A  = 3.543 = 1.13 \pi \).  The impedance \( Z(z) \) must be evaluated numerically.

\end{itemize}

To illustrate some of the differences, we are referred to fig. 5.21 \citep{pozar2009microwave}.  It is noted that

\begin{enumerate}
\item The exponential taper has the lowest cutoff frequency \( \beta L = \pi \).  Then is the Klopfenstein taper which is close \( \beta L = 1.13 \pi \).  Last is the triangular with \( \beta L = 2 \pi \).
\item The Klopfenstein taper has the lowest \( \Abs{\Gamma} \) in the passband and meets the spec of \( \Gamma_m = 0.02 \).  The worst \( \Abs{\Gamma} \) in the passband is from the exponential taper and the triangular ripple is between the two others.
\end{enumerate}
} % example
%}
