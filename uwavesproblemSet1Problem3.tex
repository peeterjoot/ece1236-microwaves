%
% Copyright � 2016 Peeter Joot.  All Rights Reserved.
% Licenced as described in the file LICENSE under the root directory of this GIT repository.
%
\makeproblem{Lossless transmission line with load.}{uwaves:problemSet1:3}{
\index{transmission!lossless loaded}

A lossless transmission line of characteristic impedance \( Z_0 = 75 \Omega \) is terminated to an
unknown complex impedance \( Z_\txtL \).  The transmission line is filled with air. The first
voltage maximum is measured at a distance \( d = 9.7 \si{mm} \) from the load and the voltage
standing-wave ratio on the line is measured to be \( \textrm{VSWR} = 2.33 \). The operating frequency
is \( f = 3.0 \si{GHz} \).

\makesubproblem{}{uwaves:problemSet1:3a}

Determine the location of the first current minimum.

\makesubproblem{}{uwaves:problemSet1:3b}

What is the magnitude (amplitude) of the reflection coefficient \( \Abs{ \Gamma_\txtL } \).

\makesubproblem{}{uwaves:problemSet1:3c}

What is the phase of the reflection coefficient \( \Theta_\txtL \)?

\makesubproblem{}{uwaves:problemSet1:3d}

Determine the load \( Z_\txtL \).

} % makeproblem

\makeanswer{uwaves:problemSet1:3}{

\makeSubAnswer{}{uwaves:problemSet1:3a}

Because the current has a different sign in the superposition sum, the current min/max will be where there is a voltage max/min, so the first minimum is at \( d = 9.7 \si{mm} \).

Showing this explicitly, the current amplitude is
\begin{equation}\label{eqn:uwavesproblemSet1Problem3:180}
\begin{aligned}
\Abs{I(-l)}
&= \Abs{ \frac{V_0^{+}}{Z_0} } \Abs{ e^{-j\beta(-l)} - \Gamma_\txtL e^{j\beta(-l)} }
\\ &= \Abs{ \frac{V_0^{+}}{Z_0} } \Abs{ 1 - \Abs{\Gamma_\txtL} e^{ j \Theta_\txtL - 2 j \beta l } }.
\end{aligned}
\end{equation}

The current minimum and maximums are respectively
\begin{equation}\label{eqn:uwavesproblemSet1Problem3:200}
\begin{aligned}
\Abs{I(-l)}_{\mathrm{min}} &= \Abs{ \frac{V_0^{+}}{Z_0} } \lr{ 1 - \Abs{\Gamma_\txtL} } \\
\Abs{I(-l)}_{\mathrm{max}} &= \Abs{ \frac{V_0^{+}}{Z_0} } \lr{ 1 + \Abs{\Gamma_\txtL} } \\
\end{aligned}
\end{equation}

and occur when \( \Theta_\txtL - 2 \beta l = 2 \pi k \), and \( \Theta_\txtL - 2 \beta l = (2k - 1)\pi \) respectively.

The first current minimum occurs when \( \Theta_\txtL = 2 \beta l \), which is exactly where the first voltage maximum also occurs.

\makeSubAnswer{}{uwaves:problemSet1:3b}

We have \( R = G = 0 \), so

\begin{equation}\label{eqn:uwavesproblemSet1Problem3:20}
Z_0 = \sqrt{\frac{L}{C}} = 75 \Omega.
\end{equation}

We also have
\begin{equation}\label{eqn:uwavesproblemSet1Problem3:40}
v_\phi = c = \inv{\sqrt{L C}}.
\end{equation}

The VSWR is

\begin{equation}\label{eqn:uwavesproblemSet1Problem3:60}
\begin{aligned}
V_{\textrm{SWR}}
&= \frac{V_{\mathrm{max}}}{V_{\mathrm{min}}}
\\ &= \frac
{1 + \Abs{\Gamma_\txtL}}
{1 - \Abs{\Gamma_\txtL}}
\\ &= 2.33,
\end{aligned}
\end{equation}

so

\begin{equation}\label{eqn:uwavesproblemSet1Problem3:80}
1 + \Abs{\Gamma_\txtL}
=
\lr{1 - \Abs{\Gamma_\txtL}} V_{\textrm{SWR}}
\end{equation}

or

\begin{equation}\label{eqn:uwavesproblemSet1Problem3:100}
\begin{aligned}
\Abs{\Gamma_\txtL}
&= \frac{V_{\textrm{SWR}} - 1}{V_{\textrm{SWR}} + 1}
\\ &= \frac{2.33 - 1}{2.33+1}
\\ &= \frac{1.33}{3.33}
\\ &= \frac{4/3}{10/3}
\\ &= 0.4 \,\Omega.
\end{aligned}
\end{equation}

\makeSubAnswer{}{uwaves:problemSet1:3c}

The absolute voltage at \( z = - l \) is

\begin{equation}\label{eqn:uwavesproblemSet1Problem3:120}
\Abs{V(-l)} = \Abs{V_0^{+}}\Abs{ 1 + \Abs{\Gamma_\txtL} e^{j(\Theta_\txtL - 2 \beta l)} },
\end{equation}

so the maximum was measured at

\begin{equation}\label{eqn:uwavesproblemSet1Problem3:160}
\Theta_\txtL - 2 \beta d = 2 \pi (0),
\end{equation}

or

\begin{equation}\label{eqn:uwavesproblemSet1Problem3:140}
\begin{aligned}
\Theta_\txtL
&= 2 \beta d
\\ &= 2 \frac{\omega}{v_\phi} d
\\ &= \frac{4 \pi f d}{c}
\\ &= \frac{4 \pi (3 \times 10^9 ) (9.7 \times 10^{-3}) }{3 \times 10^8}
\\ &= 1.219 \,\si{rad}.
\end{aligned}
\end{equation}

The reflection coefficient can now be written out explicitly

\begin{equation}\label{eqn:uwavesproblemSet1Problem3:n}
\Gamma_\L = 0.1377 + 0.3749 j \,\Omega.
\end{equation}

\makeSubAnswer{}{uwaves:problemSet1:3d}

The load impedance is

\begin{equation}\label{eqn:uwavesproblemSet1Problem3:220}
\begin{aligned}
Z_\txtL
&= Z_0 \frac{ 1 + \Gamma_\txtL }{ 1 - \Gamma_\txtL }
\\ &= 75 \frac{ 1.4 + 0.049 j }{ 0.6 - 0.049 j }
\\ &= 71.29 + 63.6 j \,\Omega.
\end{aligned}
\end{equation}

The numerical results are computed in \nbref{ps1:ps1_3.jl}.
}
