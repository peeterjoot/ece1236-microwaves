%
% Copyright � 2016 Peeter Joot.  All Rights Reserved.
% Licenced as described in the file LICENSE under the root directory of this GIT repository.
%
\makeproblem{Transmission line basics.}{uwaves:problemSet1:1}{
A two-wire line has distributed parameters \( L' = 6.5 \mu \si{ H/m} \) and \( C' = 8.7 \si{pF/m} \).
\makesubproblem{}{uwaves:problemSet1:1a}
What is the characteristic impedance (resistance) of the line?
\makesubproblem{}{uwaves:problemSet1:1b}
What is the phase velocity this transmission line?
\makesubproblem{}{uwaves:problemSet1:1c}
If a voltage wave with time waveform \( V^{-}(t)\) is traveling along the line in the \( -z\) direction, and has an amplitude of \( 10 \si{V} \), write a time-domain expression for the current wave associated with this voltage wave, using numerical parameters determined in
\partref{uwaves:problemSet1:1a} and
\partref{uwaves:problemSet1:1b}.
} % makeproblem
\makeanswer{uwaves:problemSet1:1}{
\makeSubAnswer{}{uwaves:problemSet1:1a}
In this case we have
\begin{equation}\label{eqn:uwavesproblemSet1Problem1:20}
\begin{aligned}
\gamma &= \sqrt{(j \omega L')(j \omega C')} \\ &= j \omega \sqrt{L' C'}
\end{aligned}
\end{equation}
and impedance
\begin{equation}\label{eqn:uwavesproblemSet1Problem1:40}
\begin{aligned}
Z_0
&= \frac{ \cancel{R'} + j \omega L' }{\gamma}
\\ &= \frac{L'}{\sqrt{L' C' }}
\\ &= \sqrt{\frac{L'}{C'}}
\\ &= \sqrt{ \frac{ 6.5 \mu \si{H/m}}{ 8.7 \si{p F/m} } }
\\ &= \sqrt{ \frac{ 6.5 \times 10^{-6} \si{H}}{ 8.7 \times 10^{-12} \si{F} } }
\\ &= \sqrt{ \frac{ 6.5 }{ 8.7 } } \si{k} \Omega.
\\ &= 0.86 \si{k} \Omega.
\end{aligned}
\end{equation}
\makeSubAnswer{}{uwaves:problemSet1:1b}
The phase velocity is
\begin{equation}\label{eqn:uwavesproblemSet1Problem1:60}
\begin{aligned}
v_\phi
&= \frac{1}{\sqrt{L' C'}}
\\ &= \frac{1}{\sqrt{ (6.5)(8.7) 10^{-6-12} \si{ (H/m) (F/m)} }}
\\ &= 1.3 \times 10^8 \si{m/s},
\end{aligned}
\end{equation}
a factor of 23 less than the speed of light in vacuum.

A verification that \( \sqrt{\si{H/F}} \) is an Ohm, and that (\( 1/\sqrt{\si{(H/m)(F/m)}} = \si{m/s}\) can be found in \nbref{transmissionLineUnits.nb}.
\makeSubAnswer{}{uwaves:problemSet1:1c}
The phasor voltage is
\begin{equation}\label{eqn:uwavesproblemSet1Problem1:80}
V = 10 \times e^{\beta z},
\end{equation}
so the current phasor is
\begin{equation}\label{eqn:uwavesproblemSet1Problem1:100}
I = -\frac{10 V}{0.86 k\Omega} \times e^{\beta z},
\end{equation}

We have
\begin{equation}\label{eqn:uwavesproblemSet1Problem1:120}
\begin{aligned}
\beta
&= \omega \sqrt{L C}
\\ &= \frac{\omega}{v_\phi},
\end{aligned}
\end{equation}
so
\begin{equation}\label{eqn:uwavesproblemSet1Problem1:140}
i(z,t) = -11.6 \cos\lr{ \omega( z/v_\phi + t ) } \si{mA},
\end{equation}
where \( v_\phi \) is given by \cref{eqn:uwavesproblemSet1Problem1:60}.
}
